\section{Verwendung des Frameworks Angular} \label{sec:angular}

Die Implementierung zahlreicher Lösungen für die Anforderungen der Schnittstelle wird über das Angulars-Framework und seine zahlreichen Konzepte erfolgen.
Dieser Teil wird sich auf die hauptsächlich verwendeten Konzepte und ihren Nutzen konzentrieren.

\subsection{Components}

Um die Verwaltung des Codes zu vereinfachen, schlägt Angular vor, die verschiedenen Teile einer Anwendung, die mehrfach verwendet werden, in Komponenten aufzuteilen.
Diese Komponenten bestehen hauptsächlich aus einer TypeScrypt-Klasse, mit der die Komponente gesteuert werden kann, einer dynamischen HTML-Schablone, mit der ein \ac{DOM} leicht erstellt werden kann, und CSS-Stilen, um das visuelle Erscheinungsbild zu verwalten.
Im Quellcode sind die Komponenten an den drei Dateien gemäß dem Benennungsstandard zu erkennen:

\begin{itemize}
  \item \textbf{Controller}: ComponentName.component.ts
  \item \textbf{Template}: ComponentName.component.html
  \item \textbf{Styles}: ComponentName.component.scss
\end{itemize}

Diese drei Dateien werden immer in einem Ordner mit dem Namen der Komponente zusammengefasst.
Die HTML-Schablone kann mithilfe der verschiedenen Syntax von Angular für den Controller reaktionsfähig gemacht werden\footnote{Siehe \href{https://angular.io/guide/template-syntax}{https://angular.io/guide/template-syntax}}.
So würde eine Komponente zur Anzeige eines Zählers mit einer Schaltfläche zum Inkrementieren des Zählers aussehen:

\begin{lstlisting}[
  language=javascript,
  caption={Controller der Komponente counter-increment},
  captionpos=b,
  label={lst:component_example}
]
import { Component } from "@angular/core";

@Component({
  selector: "counter-increment",
  templateUrl: "./counter-increment.component.html",
  styleUrls: ["./counter-increment.component.css"]
})
export class CounterIncrementComponent {
  public counter = 0;

  public increment = (): void => {
    this.counter = this.counter + 1;
  };
}
\end{lstlisting}

\begin{lstlisting}[language=html,caption={Template der Komponente counter-increment},captionpos=b]
<h1>{{counter}}</h1>
<button (click)="increment()">Increment</button>
\end{lstlisting}

\subsection{Router}

Angular-Komponenten ermöglichen es, Teile der Benutzeroberfläche zu isolieren, von Mikrokomponenten wie einer Schaltfläche bis hin zu den verschiedenen Seiten der Benutzeroberfläche.
Es ist wichtig, dass der Nutzer die verschiedenen Seiten über eine eindeutige \ac{URL} aufruft, wie im Abschnitt \ref{sec:workload} erwähnt.
Das Angular-Framework bietet eine Out-of-the-Box-Lösung, die es ermöglicht, eine bestimmte Komponente auf der Grundlage des \ac{URI} der Anwendung anzuzeigen, so dass auch die vom \ac{URI} übertragenen Informationen abgerufen werden können: Angular Router\footnote{Siehe \href{https://angular.io/guide/router}{https://angular.io/guide/router}}.
Der Router wird einfach mit einem Array aus mehreren Routen\footnote{Siehe \href{https://angular.io/api/router/Route}{https://angular.io/api/router/Route}} konfiguriert, das aus einem Pfad und der anzuzeigenden Komponente besteht.

\begin{lstlisting}[language=javascript,caption={Router-Einstellung, mit der zwei Komponenten an zwei URIs der Anwendung angezeigt werden},captionpos=b]
const routes: Routes = [
  { path: '', component: HomePageComponent },
  { path: 'map', component: MapPageComponent }
];
\end{lstlisting}

\subsection{Services}
Die Anwendung besteht also aus vielen verschiedenen Komponenten.
Es kann jedoch vorkommen, dass diese Komponenten ein und dasselbe Stück Code mehrfach verwenden müssen.
Angular implementiert dafür sogenannte Services.
Services sind ein Mechanismus, um die Funktionalität zwischen den Komponenten zu teilen.
Es gibt ein Prinzip in der Softwareentwicklung, nämlich \ac{DRY}, und die Verwendung von Services in Angular folgt sicher diesem Prinzip.
Die Logik oder Daten werden in einem Service implementiert und der Service kann über mehrere Komponenten Ihrer Angular-Anwendung verwendet werden.
Das typischste Beispiel für die Nutzung eines Service ist der Aufruf einer API.

\begin{lstlisting}[
  language=javascript,
  caption={Beispiel für einen Service, der die Interaktion mit einer externen API ermöglicht},
  captionpos=b,
  label={lst:service_example}
  ]
import { Injectable } from '@angular/core'
import { HttpClient } from '@angular/common/http'
import { Observable } from 'rxjs'

export class User {
  id: string;
  name: string;
  email: string;
}

@Injectable({
  providedIn: 'root'
})
export class UserService {

  endPoint = 'https://path/to/api'

  constructor(private httpClient: HttpClient) { }

  getAll(): Observable<User[]> {
    return this.httpClient.get<User[]>(this.endPoint + '/users')
  }

  get(id: number): Observable<User> {
    return this.httpClient.get<User>(this.endPoint + '/users/' + id)
  }  

  delete(id: number): Observable<User> {
    return this.httpClient.delete(this.endPoint + '/users/' + id)
  }
}
\end{lstlisting}

\begin{lstlisting}[language=javascript,caption={Vereinfachtes Beispiel für die Nutzung des Service definiert in \ref{lst:service_example}},captionpos=b]
import { UserService, User } from "./user.service"

// ...

export class UsersComponent implements OnInit {
  
  users: User[] = []

  constructor(public userService: UserService) { }

  ngOnInit() {
    this.userService.getAll().subscribe((users) => {
      this.users = users
    })    
  }  
}
\end{lstlisting}

\subsection{Decorator}

Ein Angular-Dekorator ist eine Funktion, mit der Metadaten an eine Klassendeklaration, eine Methode, einen Accessor, eine Eigenschaft oder einen Parameter angehängt werden können.
Wie bereits in den Beispielen zu sehen ist, verwendet Angular Dekoratoren, um der Klasse, der Methode oder der Eigenschaft Metadaten zur Verfügung zu stellen.
Beispielsweise müssen Komponenten wissen, wo sie ihre Vorlage finden können. Dies ist möglich, indem der Parameter \lstinline{templateUrl} des \lstinline{@Component}-Dekorators verwendet wird (siehe \ref{lst:component_example}).