\section{Ergonomie und User Experience}


Es gibt heute viele Möglichkeiten, eine Benutzeroberfläche zu präsentieren.
Viele Schnittstellen vermitteln jedoch den Eindruck, dass sie schwer zu erforschen und zu nutzen sind, und es ist schwierig, das ursprüngliche Ziel zu erreichen.
So mussten Konzepte erstellt werden, um einerseits eine Schnittstelle zu entwerfen und andererseits eine Schnittstelle, die von einem Menschen genutzt wird, so zu bewerten, dass sie effektiv, effizient und mit denen die Nutzer zufrieden sein werden.
So entstanden zwei Designkonzepte, die seither allgemein verwendet werden: Ergonomie und \ac{UX} von Schnittstellen.

\subsection{Ergonomie}

Ergonomie kann definiert werden als ``Ausmaß, in dem ein Produkt von bestimmten Benutzern verwendet werden kann, um bestimmte Ziele mit Effektivität, Effizienz und Zufriedenheit in einem bestimmten Nutzungskontext zu erreichen''\cite{usability}.
Da es sich um eine recht umfassende und allgemeine Definition handelt, kann man sie in zwei Konzepte unterteilen.

Eine der Konzeptionen ist, dass sich die Ergonomie auf Maßnahmen konzentrieren sollte, die mit der Erfüllung der übergeordneten Ziele der Aufgabe zusammenhängen. Dies beinhaltet Techniken zur summativen Bewertung oder messbasierten Bewertung, sowohl objektive als auch subjektive Schnittstellen.

Die andere Konzeption ist, dass sich die Praktiker auf die Erkennung und Beseitigung von Problemen der Benutzerfreundlichkeit konzentrieren sollten.

\subsection{User Experience}

Während die Ergonomie also technisch und wissenschaftlich orientiert ist, um einen Benutzer bei der Ausführung einer Aufgabe zu bewerten, ist die UX emotionaler.

So sind instrumentelle Attribute wie Effizienz und Effektivität immer noch wichtig, aber hauptsächlich in dem Maße, in dem sie emotionale Ergebnisse wie Zufriedenheit, Vertrauen und wahrgenommene Schönheit beeinflussen, mit daraus resultierenden Auswirkungen auf das Ergebnisverhalten wie z. B. wiederholte Käufe und Weiterempfehlung an andere Nutzer.

UX nach Geoffrey Crofte \cite{uxCrofte} ist daher ganzheitlich und zeitlich orientiert, d. h. vor, während und nach der Nutzung der Schnittstelle. Daher konzentriert sich UX auf die Faktoren, die auf die Nutzung einer Schnittstelle folgen:

\begin{itemize}
  \item das soziale Image der erwarteten Erfahrung
  \item die Erinnerung an vergangene Erlebnisse
  \item das soziale Teilen von Erfahrungen
  \item die Kontinuität des Service
  \item die Erinnerung an die durchgeführte Erfahrung
\end{itemize}