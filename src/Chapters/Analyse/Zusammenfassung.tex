\section{Zusammenfassung}

Die ergonomische Analyse, die Benutzertests und der Fragebogen für die Feuerwehr haben konkret mehr oder weniger vitale Lücken innerhalb der Silvanet-Schnittstelle aufgedeckt.
Dies führte zu einer neuen Roadmap und einer Vielzahl von Aufgaben, die erledigt werden mussten, um die Erfahrungen der vielen zukünftigen Nutzer zu verbessern.
Angesichts der neuen Arbeitsbelastung mussten wir die Aufgaben priorisieren und sie so zuweisen, dass sie in der vorgegebenen Zeit der Thesis umgesetzt werden konnten.
Eine der Hauptaufgaben wird es sein, die interaktiven Karten nutzbar zu machen und ihre \ac{SUS}-Bewertung zu verbessern.
Zweitens scheint es, dass die ergonomische Inspektion und die Benutzertests einen Bedarf für das aktuelle Navigationssystem der Benutzeroberfläche aufzeigen.
Eine Standardisierung scheint notwendig zu sein, um dem Nutzer eine einheitliche Erfahrung zu bieten.
Schließlich könnten einige gezielte Teile der Benutzeroberfläche laut dem Ergonomiebericht verbessert werden.
Dies betrifft insbesondere die Relevanz von Warnungen, die Lokalisierung der Schnittstelle und schließlich die Usability der Planungsseite einer \textit{Site}.