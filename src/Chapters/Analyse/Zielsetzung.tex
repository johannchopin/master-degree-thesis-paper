\section{Zielsetzung}

Das Ziel dieser Thesis ist die methodische Analyse der Webschnittstelle zur Steuerung eines Silvanet-Systems, um dem Benutzer eine möglichst benutzerfreundliche, effiziente und zugängliche Erfahrung zu bieten.
Der Nutzer sollte in der Lage sein, die Funktionsweise der Schnittstelle schnell und einfach zu erforschen und zu verstehen, abhängig von seinem Wissen über das System oder die Welt der Informatik im Allgemeinen und seiner Kultur.

Dazu kommt die Forschungs- und Entwicklungsarbeit zu interaktiven Karten, die in der Anwendung massiv eingesetzt werden, um Sensoren zu lokalisieren und schnell den Status eines Ortes zu erfahren.
So kann der Nutzer schnell und in Echtzeit den Status eines bestimmten Standorts oder mehrerer Standorte je nach Zoomstufe der Karte verfolgen.
Jeder Standort kann aus Tausenden von Sensoren bestehen, die über mehrere Hektar verteilt sind. Die Karte muss diese Punktdichte so verwalten, dass sie lesbar bleibt und dem Nutzer einfach die relevanten Daten liefert.

Schließlich wird eine Forschungsarbeit durchgeführt, um das Krisenmanagement zu verbessern.
Dies betrifft insbesondere das Szenario eines Waldbrandes, bei dem der Nutzer die Behörden warnen und gleichzeitig eine Überpanik vermeiden muss.
Die Schnittstelle muss daher in der Lage sein, den Benutzer schnell und effektiv über die Art der Gefahr zu informieren und ihn bei der Ausführung seiner Rolle zur Behebung der Gefahr zu leiten.