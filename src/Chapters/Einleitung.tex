\chapter{Einleitung}

\section{Motivation}

Jedes Jahr und immer häufiger brechen auf der ganzen Welt immer intensivere und verheerendere Waldbrände aus.
Nur in europa brannten 2021 mehr als 1.113.464 ha Wald in mehr als 39 Ländern \cite{JRC}.
Um das Risiko von Waldbränden zu verringern, hat das junge Start-up-Unternehmen Dryad Sensoren entwickelt, die eine ultraschnelle Erkennung von Waldbränden ermöglichen.

Diese Sensoren, die über mehrere Aktionsstandorte verteilt sind, erfassen verschiedene Arten von Daten wie Temperatur, Luftfeuchtigkeit und das Vorhandensein von Gasen, die bei einer Verbrennung entstehen.
Ein Webinterface ist verfügbar, um die verschiedenen Daten, die von diesen Sensoren gesendet werden, zu sehen.
Da die Kunden jedoch nicht alle Computerspezialisten sind, muss die Benutzeroberfläche so intuitiv wie möglich sein.

Derzeit kommt das Produkt mit einer Anleitung, die erklärt, wie man die Webanwendung nutzt. Dies deutet auf ein großes Problem mit der Nutzererfahrung hin.
Meine Aufgabe in dieser Arbeit ist es daher, die Schwachstellen der aktuellen Schnittstelle zu identifizieren und Verbesserungen vorzuschlagen und zu entwickeln.

\section{Aufgabenbeschreibung und Ziele} \label{sec:targets}

In meiner Thesis bei Dryad geht es darum, die entscheidenden Bereiche der Webanwendung, die die verschiedenen Sensoren verwaltet, zu erkennen, die als schlechte Benutzererfahrung angesehen werden.
Die Schwerpunkte der Arbeit lassen sich wie folgt beschreiben:

\begin{itemize}
  \item Identifizierung von ergonomischen Mängeln basierend auf einer Auditierung der gesamten Schnittstelle.
  \item Entwicklung und Durchführung von Benutzertests, die es ermöglichen, Tickets so zu definieren, dass sie nach Prioritäten geordnet werden können.
  \item Forschung und Entwicklung zu den verschiedenen interaktiven Karten der Anwendung, die es dem Nutzer erleichtern, seine \textit{Sites} und Sensoren sowie deren Status zu entdecken.
  \item Forschung zur Verbesserung der Datenpräsentation in Stresssituationen
\end{itemize}

\section{Aufbau der Thesis}

In Kapitel \ref{chap:grundlagen} werden zunächst die grundlegenden Konzepte vorgestellt, die für das Verständnis der These hilfreich sind.
In Kapitel \ref{chap:analyse} wird die Analyse der Lücken und Bedürfnisse der Schnittstelle weiterentwickelt, sodass in Kapitel \ref{chap:konzeption} Verbesserungen und Prototypen vorgeschlagen werden können.
Diese werden priorisiert und die wichtigsten vollständig oder teilweise in Kapitel \ref{chap:implementation} implementiert.
Der Nutzen der Implementierungen wird in Abschnitt \ref{chap:evaluation} bewertet.
In Kapitel \ref{chap:conclusion} wird schließlich ein Ausblick auf mögliche Verbesserungen der Thesis gegeben.
