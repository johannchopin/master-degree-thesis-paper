%*******************************************************
% Abstract
%*******************************************************
\pdfbookmark[0]{Zusammenfassung}{Zusammenfassung}
\chapter*{Zusammenfassung}

Eine neue Webanwendung, die die Verwaltung und Visualisierung von tausenden von Sensoren auf der ganzen Welt ermöglicht, kann für neue Nutzer kompliziert zu erlernen sein.
Dies ist oft der Fall, wenn die Bedürfnisse der Endnutzer nicht früh genug in die Produktentwicklung einbezogen werden.
Es gibt jedoch zahlreiche Techniken, mit denen man die Usability- oder Ergnonomiedefizite einer Schnittstelle bewerten und gleichzeitig Verbesserungsvorschläge machen kann.
Dieser Beitrag befasst sich mit der praktischen Anwendung dieser Techniken auf die Dryad-Schnittstelle und der Umsetzung einiger Verbesserungsideen, um festzustellen, ob die Schnittstelle eine vereinfachte Benutzbarkeit aufweist oder nicht.
